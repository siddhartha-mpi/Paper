\documentclass[twocolumn,aps,showpacs,superscriptaddress,longbibliography]{revtex4-1}
%\documentclass[preprint,aps,showpacs,superscriptaddress,longbibliography]{revtex4-1}

\usepackage{bm}
\usepackage{amsmath}
\usepackage{amssymb}
\usepackage[usenames,dvipsnames]{color}
\usepackage{graphicx}
\usepackage{dcolumn}
\usepackage{hyperref}
\usepackage{natbib}
\usepackage[caption=false]{subfig}
\usepackage{float}
\usepackage{times}
%\usepackage{pdfpages}
\usepackage{todonotes}
\hypersetup{
%  colorlinks=false,
  colorlinks=true,
  citecolor=blue,
  linkcolor=blue,
  urlcolor=blue}


\begin{document}

\title{Correlation effects in enhanced-ionization of diatomic molecules :
       A three-dimensional time-dependent generalized-active-space 
       configuration-interaction study}

\author{S.~Chattopadhyay}
\affiliation{Department of Physics and Astronomy,
             Aarhus University, DK-8000 Aarhus C, Denmark}
%\email{siddhartha.chattopadhyay@mpi-hd.mpg.de}
%\affiliation{Max-Planck-Institut f\"ur Kernphysik,
%             Saupfercheckweg 1, D-69117 Heidelberg, Germany}
\author{H.~R.~Larsson}
\affiliation{Institut f\"ur Physikalische Chemie, 
             Christian-Albrechts-Universit\"at zu Kiel, D-24098 Kiel, Germany}
\author{L.~K.~S{\o}rensen}
\affiliation{Department of Chemistry-{\AA}ngstr\"om Laboratory, 
             Uppsala University, SE-75120 Uppsala, Sweden}
\author{S.~Bauch}
\affiliation{Institut f\"ur Theoretische Physik und Astrophysik,
             Christian-Albrechts-Universit\"at zu Kiel, D-24098 Kiel, Germany}
\author{L.~B.~Madsen}
\affiliation{Department of Physics and Astronomy,
             Aarhus University, DK-8000 Aarhus C, Denmark}


\date{\today}

%%%%%%%%%%%%%%%%%%%%%%%%%%%%%%%%%%%%%%%%%%%%%%%%%%%%%%%%%%%%%%%%%%%%%%%%%%%%%%%%%%%%%%%%%%%%%%%%%%%%
%%%%%%%                                  Abstract                                             %%%%%%
%%%%%%%%%%%%%%%%%%%%%%%%%%%%%%%%%%%%%%%%%%%%%%%%%%%%%%%%%%%%%%%%%%%%%%%%%%%%%%%%%%%%%%%%%%%%%%%%%%%%
\begin{abstract}

\end{abstract}

%\pacs{31.15.-p, 32.80.Fb, 33.80.Eh, 42.50.Hz}

\maketitle

%%%%%%%%%%%%%%%%%%%%%%%%%%%%%%%%%%%%%%%%%%%%%%%%%%%%%%%%%%%%%%%%%%%%%%%%%%%%%%%%%%%%%%%%%%%%%%%%%%%%
%%%%%%%                              Section: Introduction                                    %%%%%%
%%%%%%%%%%%%%%%%%%%%%%%%%%%%%%%%%%%%%%%%%%%%%%%%%%%%%%%%%%%%%%%%%%%%%%%%%%%%%%%%%%%%%%%%%%%%%%%%%%%%
\section{Introduction}

%%%%%%%%%%%%%%%%%%%%%%%%%%%%%%%%%%%%%%%%%%%%%%%%%%%%%%%%%%%%%%%%%%%%%%%%%%%%%%%%%%%%%%%%%%%%%%%%%%%%
%%%%%%%                                  Section: Theory                                      %%%%%%
%%%%%%%%%%%%%%%%%%%%%%%%%%%%%%%%%%%%%%%%%%%%%%%%%%%%%%%%%%%%%%%%%%%%%%%%%%%%%%%%%%%%%%%%%%%%%%%%%%%%
\section{Theory}
\label{theory}

%%%%%%%%%%%%%%%%%%%%%%%%%%%%%%%%%%%%%%%%%%%%%%%%%%%%%%%%%%%%%%%%%%%%%%%%%%%%%%%%%%%%%%%%%%%%%%%%%%%%
%%%%%%%                                  Section: Numerical examples                          %%%%%%
%%%%%%%%%%%%%%%%%%%%%%%%%%%%%%%%%%%%%%%%%%%%%%%%%%%%%%%%%%%%%%%%%%%%%%%%%%%%%%%%%%%%%%%%%%%%%%%%%%%%
\section{Correlation effects in enhanced ionization}
\label{numerical_ex}


%%%%%%%%%%%%%%%%%%%%%%%%%%%%%%%%%%%%%%%%%%%%%%%%%%%%%%%%%%%%%%%%%%%%%%%%%%%%%%%%%%%%%%%%%%%%%%%%%%%%
%%%%%%%                   Subsection: Two electron H_2 molecule                               %%%%%%
%%%%%%%%%%%%%%%%%%%%%%%%%%%%%%%%%%%%%%%%%%%%%%%%%%%%%%%%%%%%%%%%%%%%%%%%%%%%%%%%%%%%%%%%%%%%%%%%%%%%
\subsection{Two-electron $\mathrm{H}_2$ molecule}
\label{h2}
\begin{figure}
 \includegraphics[width=\columnwidth]{./plots/H2-GS}
 \caption{(color online). Ground-state energy of the  $\mathrm{H}_2$-molecule from Imaginary time
           propagation.
           In the lower inset we enlarge a small region around the equilibrium inter-nuclear separation
           to demonstrate the convergence of different GAS methods.
           CAS$^*(2,3)$ represents the TD-GASCI calculation with three active orbitals in the GAS.
           Similarly  CAS$^*(2,12)$ represents the TD-GASCI calculation with twelve active orbitals
           in the GAS.}
 \label{fig:H2-binding}
\end{figure}

\begin{figure}
 \includegraphics[width=\columnwidth]{./plots/H2-comp}
 \caption{(color online). Total ground-state energy of the 2-electron $\mathrm{H}_2$-like molecule.
           The upper inset shows that the SAE and the CIS approximations overlap with the HF result.
           In the lower inset we enlarge a small region around the equilibrium inter-nuclear separation
           to demonstrate the convergence of different GAS methods.
           CAS$^*(2,3)$ represents the TD-GASCI calculation with three active orbitals in the GAS.
           Similarly  CAS$^*(2,12)$ represents the TD-GASCI calculation with twelve active orbitals
           in the GAS.}
 \label{fig:H2-comparison}
\end{figure}

\begin{figure}
 \includegraphics[width=\columnwidth]{./plots/H2-EI-053}
 \caption{(color online). Total ground-state energy of the 2-electron $\mathrm{H}_2$-like molecule.
           The upper inset shows that the SAE and the CIS approximations overlap with the HF result.
           In the lower inset we enlarge a small region around the equilibrium inter-nuclear separation
           to demonstrate the convergence of different GAS methods.
           CAS$^*(2,3)$ represents the TD-GASCI calculation with three active orbitals in the GAS.
           Similarly  CAS$^*(2,12)$ represents the TD-GASCI calculation with twelve active orbitals
           in the GAS.}
 \label{fig:H2-CAS}
\end{figure}


\begin{figure}
    \includegraphics[width=\columnwidth]{./plots/H2-ES-1}
    \caption{Field-free ground state and the three lowest lying excited states of model-$\mathrm{H}_2$ :
             (a) comparison between the CAS$^*(2,2)$ and CAS$^*(2,3)$ models,
             (b) comparison between the CAS$^*(2,3)$ and CAS$^*(2,6)$ models.
              The CAS$^*(2,3)$ model energies overlap with CAS$^*(2,6)$ model energies.
              The $i$-in the notation CAS$^*(2,N)$-$i$ labels the state starting with $i = 1$ for
              the ground state.}
     \label{fig:H2-es}
\end{figure}

\begin{figure}
    \includegraphics[width=\columnwidth]{./plots/H2-ES-2}
    \caption{Field-free ground state and the three lowest lying excited states of model-$\mathrm{H}_2$ :
             (a) comparison between the CAS$^*(2,2)$ and CAS$^*(2,3)$ models,
             (b) comparison between the CAS$^*(2,3)$ and CAS$^*(2,6)$ models.
              The CAS$^*(2,3)$ model energies overlap with CAS$^*(2,6)$ model energies.
              The $i$-in the notation CAS$^*(2,N)$-$i$ labels the state starting with $i = 1$ for
              the ground state.}
     \label{fig:H2-es}
\end{figure}




%%%%%%%%%%%%%%%%%%%%%%%%%%%%%%%%%%%%%%%%%%%%%%%%%%%%%%%%%%%%%%%%%%%%%%%%%%%%%%%%%%%%%%%%%%%%%%%%%%%%
%%%%%%%                        Subsection: Four electron LiH molecule                         %%%%%%
%%%%%%%%%%%%%%%%%%%%%%%%%%%%%%%%%%%%%%%%%%%%%%%%%%%%%%%%%%%%%%%%%%%%%%%%%%%%%%%%%%%%%%%%%%%%%%%%%%%%
\subsection{Four-electron $\mathrm{LiH}$ molecule}
\label{lih}
\begin{figure}
 \includegraphics[width=\columnwidth]{./plots/LiH-EI-CAS}
 \caption{(color online). Total ground-state energy of the $\mathrm{H}_2$ molecule.
           to demonstrate the convergence of different GAS methods.
           CAS$^*(2,3)$ represents the TD-GASCI calculation with three active orbitals in the GAS.
           Similarly  CAS$^*(2,12)$ represents the TD-GASCI calculation with twelve active orbitals
           in the GAS.}
 \label{fig:H2-CAS}
\end{figure}




%%%%%%%%%%%%%%%%%%%%%%%%%%%%%%%%%%%%%%%%%%%%%%%%%%%%%%%%%%%%%%%%%%%%%%%%%%%%%%%%%%%%%%%%%%%%%%%%%%%%
%%%%%%%                        Subsection: Ten electron HF molecule                           %%%%%%
%%%%%%%%%%%%%%%%%%%%%%%%%%%%%%%%%%%%%%%%%%%%%%%%%%%%%%%%%%%%%%%%%%%%%%%%%%%%%%%%%%%%%%%%%%%%%%%%%%%%
\subsection{$\mathrm{HF}$ molecule}
\label{hf}

%%%%%%%%%%%%%%%%%%%%%%%%%%%%%%%%%%%%%%%%%%%%%%%%%%%%%%%%%%%%%%%%%%%%%%%%%%%%%%%%%%%%%%%%%%%%%%%%%%%%
%%%%%%%%%%%%%                        Section: Conclusion                                      %%%%%%
%%%%%%%%%%%%%%%%%%%%%%%%%%%%%%%%%%%%%%%%%%%%%%%%%%%%%%%%%%%%%%%%%%%%%%%%%%%%%%%%%%%%%%%%%%%%%%%%%%%%
\section{Summary and Conclusion}
\label{conc}

%%%%%%%%%%%%%%%%%%%%%%%%%%%%%%%%%%%%%%%%%%%%%%%%%%%%%%%%%%%%%%%%%%%%%%%%%%%%%%%%%%%%%%%%%%%%%%%%%%%%
%%%%%%%%%%%%                               Acknowledgements                           %%%%%%%%%%%%%%
%%%%%%%%%%%%%%%%%%%%%%%%%%%%%%%%%%%%%%%%%%%%%%%%%%%%%%%%%%%%%%%%%%%%%%%%%%%%%%%%%%%%%%%%%%%%%%%%%%%%
\begin{acknowledgments}
This work was supported by the ERC-StG (Project No. 277767-TDMET), the VKR center of excellence, 
QUSCOPE, and the BMBF in the frame of the ``Verbundprojekt FSP 302''

\end{acknowledgments}
%%%%%%%%%%%%%%%%%%%%%%%%%%%%%%%%%%%%%%%%%%%%%%%%%%%%%%%%%%%%%%%%%%%%%%%%%%%%%%%%%%%%%%%%%%%%%%%%%%%%%
%%%%%%%%%%%%                                 References                                %%%%%%%%%%%%%%
%%%%%%%%%%%%%%%%%%%%%%%%%%%%%%%%%%%%%%%%%%%%%%%%%%%%%%%%%%%%%%%%%%%%%%%%%%%%%%%%%%%%%%%%%%%%%%%%%%%%%
\bibliography{references}{}


\end{document}
